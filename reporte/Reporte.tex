%! Author = vladimir
%! Date = 05/02/21

% Preamble
\documentclass[12pt]{article}

% Packages
\usepackage{amsmath}
\usepackage[spanish,activeacute]{babel}
\usepackage{setspace}
%---------------- Marco del doc y separacion entre lineas -----------------
\usepackage[a4paper]{geometry}
\geometry{top= 1 in,bottom = 1 in , left = 1 in, right = 1 in}
\doublespacing
%\geometry{top=2cm, bottom=2.0cm, left=2.5cm, right=2cm}
% \linespread{1.3}


%----- Para no poner sangria ----------------
\setlength{\parindent}{0cm}


% Title
\title{The Knigh's Tour Problem}
\author{Sierra Casiano Vladimir}

% Document
\begin{document}

    \maketitle

    \begin{abstract}
        Para el presente proyecto se abordar'a una variante del Problema del Recorrido del
        Caballo.
        Se presentar'an dos maneras para encontrar soluciones
        'optimas, la primera es utilizando la regla de Warnsdorff, y la segunda
        es adapatando el algoritmo bioinspirado Artifitial Bee Colony.
    \end{abstract}



    


    \section{Preliminares.}

    \subsection{The Knight's Tour Problem.}


    Un problema muy estudiado a lo largo de la historia es el del Tour del caballero.
    Muchas soluciones han sido propuestas con diferentes enfoques. En el presente
    documento se presentan dos maneras diferentes de resolverlo, utilizando los dos paradigmas de
    la programaci'on declarativa.

    \subsection{Definici'on del problema.}
    La pieza del caballo en el ajedrez tiene un movimiento `en L' \; como
    se ilustra a continuaci'on.

    El problema que queremos resolver es, dada una posici'on inicial y un tama'no del tablero,
    encontrar la distancia m'as larga de un camino v'alido del caballo.

    \subsection{Algoritmos Gen'eticos.}
    Los algoritmos gen'eticos est'an inspirados en el comportamiento de
    agentes biol'ogicos.

    El algoritmo Bee Colony fue propiesto en el a'no por el investigador InvestigadorName,
    y el funcionamiento intuitivo es el siguiente.

    \begin{itemize}
        \item Inicializar
        \item Generar fuentes de comida
        \item Dada la funci'on tomar la m'as opcima.
    \end{itemize}
    \vspace{4mm}

    Los componentes para Bee Colony son:
    \begin{itemize}
        \item Una fuente de comida representa una solici'on posible para el problema.
        \item La funci'on fitness, que evalua la calidad de una fuente de comida.
    \end{itemize}

    \textbf{Metodolog'ia}
    \begin{itemize}
        \item El n'umero de employed bees es igual al n'umero de fuentes de comida.
        \item Se generan K fuentes de comida de manera aleatoria.
    \end{itemize}


    \subsection{Ventajas de usar Programaci'on Declarativa.}

    \section{Soluci'on.}

    \subsection{Soluci'on con Prolog.}
    La primera soluci'on propuesta es utilizando una b'usqueda BFS.
    El 'arbol de posibles movimientos y el backtracking es manejado
    gracias al no determinismo de Prolog.

    Los movimientos est'an definidos por la forma `L' \; por lo tanto, deben de cumplir
    que
    \begin{itemize}
        \item La distancia entre
        \item Los valores nuevos est'an en un rango entre 1 y n.
    \end{itemize}

    Con prolog hacemos una b'usqueda DFS, y al final tomamos
    la longitud mayor de alguno de los caminos.

    \subsection{Soluci'on con Bee Colony.}

    La segunda soluci'on tiene la siguiente adaptaci'on.
    Los 8 posibles movimientos se identificar'an con un n'umero:
    \begin{itemize}
        \item Movimiento 1.
        \item Movimiento 2.
    \end{itemize}

    Los movimientos ser'an secuencias de estos 8 identificadores.
    Por ejemplo, la secuencia 1, 3, 6 corresponde a.
    \section{Implementaci'on}
    
    \section{Conclusiones.}

    \section{Trabajo futuro.}


\end{document}