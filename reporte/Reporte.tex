%! Author = vladimir
%! Date = 05/02/21

% Preamble
\documentclass[12pt]{article}

% Packages
\usepackage{amsmath}
\usepackage[spanish,activeacute]{babel}
\usepackage{setspace}
\usepackage{biblatex}
%---------------- Marco del doc y separacion entre lineas -----------------
\usepackage[a4paper]{geometry}
\geometry{top= 1 in,bottom = 1 in , left = 1 in, right = 1 in}
\doublespacing
%\geometry{top=2cm, bottom=2.0cm, left=2.5cm, right=2cm}
% \linespread{1.3}

%--- bibiliografia
\addbibresource{bibliography/Anan.bib}
\addbibresource{bibliography/Gulumbic2021.bib}
\addbibresource{bibliography/Watkins.bib}


%----- Para no poner sangria ----------------
\setlength{\parindent}{0cm}


% Title
\title{Solución al Problema del Recorrido del Caballo
    utilizando Artifitial Bee Colony.}
\author{Sierra Casiano Vladimir}

% Document
\begin{document}

    \maketitle

    \begin{abstract}
        Para el presente proyecto se abordar'a una variante del Problema del Recorrido del
        Caballo.
        Se presentar'an dos maneras para encontrar soluciones
        'optimas, la primera es utilizando la regla de Warnsdorff, y la segunda
        es adapatando el algoritmo bioinspirado Artifitial Bee Colony. Ambas
        soluciones tendr'an un enfoque declarativo.
    \end{abstract}


    \section{Preliminares.}

    \subsection{Problema del Recorrido del Caballo.}


    El problema data a mediados del Siglo VI en la India \cite{watkins} (muy cerca del
    origen del ajedrez) y ha sido muy estudiado a lo largo de la historia por
    celebres matem'aticos como
    Leonard Euler, quien enunci'o el problema de la siguiente manera:
    \textit{ ?`C'omo se pueden encontrar todas las secuencias de movimientos de
    la pieza del caballo en un tablero de ajedrez de tal manera que cada casilla del
    tablero es visitada exactamente una vez?
    }  \cite{golumbic2021} \\
    El problema es una instancia del m'as general
    Problema del camino Hamiltoniano de la  Teor'ia de Gr'aficas.

    A lo largo del tiempo varios enfoques han sido propuestas para
    encontrar soluciones tales como la fuerza bruta, divide y vencer'as,
    b'usqueda DFS con backtracking e incluso redes neuronales.
    El problema es NP-completo, por lo que encontrar algoritmos eficientes
    para dar soluciones sigue siendo un reto interesante.



    \subsection{Definici'on del problema.}
    La pieza del caballo en el ajedrez tiene un movimiento `en L' \; como
    se ilustra a continuaci'on.



    El problema que queremos resolver es, dada una posici'on inicial y el tama'no del tablero,
    encontrar el camino m'as largo posible utilizando los movimientos del caballo
    y visitando cada casilla a lo m'as una vez.


    \subsection{Algoritmos Gen'eticos.}
    Los algoritmos gen'eticos est'an inspirados en el comportamiento de
    agentes biol'ogicos.

    El algoritmo Bee Colony fue propiesto en el a'no por el investigador InvestigadorName,
    y el funcionamiento intuitivo es el siguiente.

    \begin{itemize}
        \item Inicializar
        \item Generar fuentes de comida
        \item Dada la funci'on tomar la m'as opcima.
    \end{itemize}
    \vspace{4mm}

    Los componentes para Bee Colony son \cite{anan} :
    \begin{itemize}
        \item Una fuente de comida representa una solici'on posible para el problema.
        \item La funci'on fitness, que evalua la calidad de una fuente de comida.
    \end{itemize}

    \textbf{Metodolog'ia}
    \begin{itemize}
        \item El n'umero de employed bees es igual al n'umero de fuentes de comida.
        \item Se generan K fuentes de comida de manera aleatoria.
    \end{itemize}


    \subsection{Ventajas de usar Programaci'on Declarativa.}

    \section{Soluci'on.}

    \subsection{Soluci'on con Prolog.}
    La primera soluci'on propuesta es utilizando una b'usqueda BFS.
    El 'arbol de posibles movimientos y el backtracking es manejado
    gracias al no determinismo de Prolog.

    Los movimientos est'an definidos por la forma `L' \; por lo tanto, deben de cumplir
    que
    \begin{itemize}
        \item La distancia entre
        \item Los valores nuevos est'an en un rango entre 1 y n.
    \end{itemize}

    Con prolog hacemos una b'usqueda DFS, y al final tomamos
    la longitud mayor de alguno de los caminos.

    \subsection{Soluci'on con Bee Colony.}

    La segunda soluci'on tiene la siguiente adaptaci'on.
    Los 8 posibles movimientos se identificar'an con un n'umero:
    \begin{itemize}
        \item Movimiento 1.
        \item Movimiento 2.
    \end{itemize}

    Los movimientos ser'an secuencias de estos 8 identificadores.
    Por ejemplo, la secuencia 1, 3, 6 corresponde a.
    \section{Implementaci'on}
    
    \section{Conclusiones.}

    \section{Trabajo futuro.}

    \section{Bibliograf'ia}
    \printbibliography

\end{document}